\chapter{How to write in \LaTeX}

\section{Important Stuff}

This section contains some important information about writing in \LaTeX.

\subsection{Notes about Quotes and Internal references}

When you quote something in \LaTeX, you have to do it like ``this.''
If you do it like ''this,'' one of your quotation marks will be backwards.

Another nice thing about using \LaTeX \ is that you can reference 
equations and figures once you have labeled them. For example, \cref{fig:universe} is a figure. You can also 
reference equations, sections, lemmas, theorems, etc. See the
examples in \cref{sec:moremath}.



\subsection{More Math Stuff}
\label{sec:moremath}

Mathematical expressions, like $x^{2} + 3$, are easy to create in \LaTeX.
It is also easy to have a formula written on a separate line, like \[x^{2} + 3 = 0.\]



Numbered equations are easy too!
\begin{equation}
	\label{eq:simpleequation}
	x^{5} - 9x^{4} = \sin 3.
\end{equation}
You can also refer to previous equations.
The formula
\begin{equation}
	x = \frac{-b \pm \sqrt{b^{2} - 4ac}}{2a} \label{yourlabel}
\end{equation}
is the quadratic formula.
You can use \cref{yourlabel} to solve quadratic equations.

% Notice that you can also place equations and formulas between the commands 
% ''\begin{equation}'' and ''\end{equation}''. No dollar signs are necessary in this case. 

% You can also create a label for the equation using the command ''\label''. You can 
% refer back to the equation using the ''\ref'' command. Notice how LaTeX 
% automatically takes care of the numbering. When you use the ''\ref'' command, you 
% will need to run LaTeX on the source file twice. This is necessary so that LaTeX 
% can keep track of the numbering. (Note that LaTeX will complain a little bit the first 
% time you run the source file through. This is not serious.)

Here is a formula with some simple mathematical operations:
\[5 \times 3 + 4 \div 2 = 2 \cdot 10 - 3 < 100.\]
Another simple formula:
\[x^{2} \geq 0.\]		% The command ''\geq'' is ''greater than or equal'' and 
% ''\leq'' is ''less than or equal''.				
This formula is true for $x \in \mathbb{R}$.	% The command ''\mathbb'' specifies 
% a certain type of math font. There are 
% also the ''\mathbf'' (bold) and the 
% ''\mathcal'' (calligraphic) commands, 
% among others, for specifying font styles 
% in math formulas.
One more simple formula:
\[0 \neq \frac{\pi}{2}.\]
Some simple trigonometry:
\[\sin(2x) = 2 \sin x \cos x.\]
Some simple logarithms:
\[\log(x) = \frac{\ln x}{\ln 10}.\]


And now some Greek letters:
\[\alpha + \beta = \gamma.\]
A formula involving set notation:
\[A \cup (B \cap C) = (A \cup B) \cap (A \cup C).\]
The derivative:
\[\lim_{h \rightarrow 0}\frac{f(x + h) - f(x)}{h} = f'(x).\]
The Riemann sum
\[\sum_{i = 1}^{n}f(x_{i})\Delta x\]
approximates the integral
\[\int_{a}^{b}f(x)\;dx.\]		% The characters ``\;'' just place some additional space 
% in the formula.
Finally, an infinite series:
\[1, \frac{1}{2}, \frac{1}{4}, \ldots.\] %Also try \cdots





\section{Conclusion}
``I always thought something was fundamentally wrong with the universe.'' \cite{adams1995hitchhiker}

