\chapter{How to use this Worksheet}

\section{First Section}

Uses kpfonts.

\begin{defn}[Definition Title]{label_defn1}
	A definition. The title is optional.
\end{defn}

Typewriter font is written as {\tt Computer Code}.

\begin{exmpl}[Title]{label_exmpl}
	\begin{itemize}
		\item First example 
		\begin{itemize}
			\item  Note $1$
			\item  Note $2$
		\end{itemize}
		\item Second example
	\end{itemize}
\end{exmpl}    



Some other text.

\begin{theo}[Theorem Title]{label_thm}
	A Theorem that uses objects defined in \cref{label_defn1}. Title is optional. 
	\[\dint_a^b u v' \dx = [u(x)v(x)] \eval_a^b - \dint_a^b u'(x) v(x) \dx\]
\end{theo}


\begin{prf}{label_thm}
	Proof text ends in qed symbol. Must be supplied the label of the theorem.
\end{prf}


\question[Automatic Question numbering]{\label{question_label}
	\begin{enumerate}
		\item Greek letters and uppercase math letters are always upright. Such as \label{step_a}
		\begin{enumerate}
			\item  $\alpha x^2 +\beta$ \label{substep_a_i} 
			\item  $y =mx+ C \in \R$ 
		\end{enumerate}
		
		\item Do the  following tasks after \ref{step_a}.
		\begin{tasks}(3)
			\task first do \ref{substep_a_i}.
			\task ??    
			\task Profit
		\end{tasks}
\end{enumerate}}


\begin{solution}
	Solution to the exercise. Uncomment the ``solutionfalse'' flag to make solutions invisible.
\end{solution}


\begin{note}[Fun Fact:]
	This is a note with a custom title. Default is `Note:'.
\end{note}



\section{Second Section}

\begin{axiom}{label_axiom}
	This is an axiom.
\end{axiom}



\begin{digression}
	A digression into tangentially related topics.
\end{digression}



\subsection{Case 1: First subsection}
When  \cref{label_exmpl} is true.
\begin{warning}[A Warning:] %default is Warning:
	\lipsum[1][1-2]
\end{warning}

\subsection{Case 2: Second subsection}
Do \cref{question_label} first.



\begin{code}[Code:]
	\begin{verbatim}
		$ chmod +x hello.py
		$ ./hello.py
		
		Hello World!
	\end{verbatim}
\end{code}
